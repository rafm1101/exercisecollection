\documentclass{exercisesheet}

%%%%%%%%%%%%%%%%%
% Course, default values provided
\setVeranstaltung{Euclidean geometry}
\setSemester{Winter term 2022}
%\setVerein{Mouseion at Alexandria}
%\setverein{Musaeum}
%\setSektion{Faculty of Geometry}
%\setsektion{Institute of Geometry}
\setDozent{Aristarch}
\setAssistent{Arthur-Dieter}
\setBlattnummer{1}
%%%%%%%%%%%%%%%%%

%%%%%%%%%%%%%%%%%
% load exercise collection package with the style definitions of exercises
\usepackage[blatt]{exercisecollection}
%%%%%%%%%%%%%%%%%

%%%%%%%%%%%%%%%%%
%\hypersetup{colorlinks=true, urlcolor=blue}
%%%%%%%%%%%%%%%%%

\begin{document}

%%%%%%%%%%%%%
% print header and title
\Kopf
\Titel[Exercises]<published: 27.04. -- due to 11.05.>
%%%%%%%%%%%%%

%%%%%%%%%%%%%
% if any contents follows at the end, initialise temporary files
% needs to be called after \includesolutions and \includematerials if further separate solutions/materials need to be collected
\IniFiles
%%%%%%%%%%%%%

%%%%%%%%%%%%%
% Preferences, the keys are
%	- extype: style of exercise, can be changed at any point (default: exc)
%	- points: show points (true/false)
%	- solution: where to print solutions (immediate/separate/none/info)
%	- solutionname: printed name
%	- sketch: where to print sketches of solutions (immediate/separate/none/info)
%	- sketchname: printed name
%   - solutiontitle: printed title for the separate solutions and sketches
%   - addmaterialname: printed name
%   - addmaterialtitle: printed title for the separate additional material
\SetExCollProp{extype=tutorialexercise,
		sketch=separate,
		solution=separate,
		points=false,
		solutionname=Solution~of,
		sketchname=Sketch~of,
		solutiontitle=Solutions,
		addmaterialtitle=Additional~material,
		addmaterialname=Ad}
% preferences can be changes at any place
%%%%%%%%%%%%%

%%%%%%%%%%%%%
% load any number of repositories, hand over a unique id and name of file/path to file
\LoadExerciseFile{elem01}{exc-the-elements-01}
%\LoadExerciseFile{elem02}{../repo/exc-the-elements-02}
%%%%%%%%%%%%%

%%%%%%%%%%%%%
% print exercises, use the id of the repository from above and the exercise id
\PrintExercise{elem01}{equi-tri}
\PrintExercise{elem01}{unique-id}
%%%%%%%%%%%%%

%\newpage

%%%%%%%%%%%%%
% print another block of exercises
\SetExCollProp{extype=homeworkstar,points=true}
\SetExCollProp{sketch=none, solution=none}
\PrintExercise{elem01}{unique-id}
%%%%%%%%%%%%%



%%%%%%%%%%%%%
% print optional exercises
% \emph{The following exercises are optional. Any achieved reward is added to your score.}
% \SetExCollProp{extype=hausaufgabestern}
% \PrintExercise{einstieg}{nachbarin}
%%%%%%%%%%%%%


%%%%%%%%%%%%%
% include the additiona material for these exercises
\includematerials
%%%%%%%%%%%%%

%%%%%%%%%%%%%
% Print separate solutions - unnecessary if preferences for solutions/sketches is none/info/immediate
 \newpage
 \includesolutions
%%%%%%%%%%%%%

\end{document}